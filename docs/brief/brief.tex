% !TEX TS-program = pdflatex
% !TEX encoding = UTF-8 Unicode

% This is a simple template for a LaTeX document using the "article" class.
% See "book", "report", "letter" for other types of document.

\documentclass[11pt]{article} % use larger type; default would be 10pt


\usepackage[utf8]{inputenc} % set input encoding (not needed with XeLaTeX)

%%% Examples of Article customizations
% These packages are optional, depending whether you want the features they provide.
% See the LaTeX Companion or other references for full information.

%%% PAGE DIMENSIONS
\usepackage{geometry} % to change the page dimensions
\geometry{a4paper} % or letterpaper (US) or a5paper or....
% \geometry{margin=2in} % for example, change the margins to 2 inches all round
% \geometry{landscape} % set up the page for landscape
%   read geometry.pdf for detailed page layout information

\usepackage{graphicx} % support the \includegraphics command and options

% \usepackage[parfill]{parskip} % Activate to begin paragraphs with an empty line rather than an indent


%%% PACKAGES
\usepackage{booktabs} % for much better looking tables
\usepackage{array} % for better arrays (eg matrices) in maths
\usepackage{paralist} % very flexible & customisable lists (eg. enumerate/itemize, etc.)
\usepackage{verbatim} % adds environment for commenting out blocks of text & for better verbatim
\usepackage{subfig} % make it possible to include more than one captioned figure/table in a single float
% These packages are all incorporated in the memoir class to one degree or another...

%%% HEADERS & FOOTERS
\usepackage{fancyhdr} % This should be set AFTER setting up the page geometry
\pagestyle{fancy} % options: empty , plain , fancy
\renewcommand{\headrulewidth}{0pt} % customise the layout...
\lhead{}\chead{}\rhead{}
\lfoot{}\cfoot{\thepage}\rfoot{}

%%% SECTION TITLE APPEARANCE
\usepackage{sectsty}
\allsectionsfont{\sffamily\mdseries\upshape} % (See the fntguide.pdf for font help)
% (This matches ConTeXt defaults)

%%% ToC (table of contents) APPEARANCE
\usepackage[nottoc,notlof,notlot]{tocbibind} % Put the bibliography in the ToC
\usepackage[titles,subfigure]{tocloft} % Alter the style of the Table of Contents
\renewcommand{\cftsecfont}{\rmfamily\mdseries\upshape}
\renewcommand{\cftsecpagefont}{\rmfamily\mdseries\upshape} % No bold!

%%% END Article customizations

%%% The "real" document content comes below...

\title{COMP3020 - Individual Research Project \\ Project Brief
}
\author{Student - Edward Seabrook \\ Supervisor - Tim Chown}
\date{} 


\begin{document}
\maketitle

\section*{Security of the Internet of Things}

\subsection*{Problem}
Over the years, we have noticed a huge increase in the number and type of
devices connected to the Internet. As more of our life goes online, we must
remain vigilant about protecting our personal information. A lot has been
written about the Internet of Things, its use cases and the potential
implications it will have. Ensuring that these devices are secure against
attackers is a particularly interesting topic, and if done incorrectly could
have catastrophic consequences. 

\subsection*{Goals}
The main aim of this project is to produce a thorough review of the current and
future threats that face increasingly complex networks, and the technologies
that can be used to defend against these threats. 

\subsection*{Scope}
In this project I take the Internet of Things to cover all networks consisting
of a large number of connected devices, these networks may also have global
addressability. This field may also be referred to as ubiquitous or pervasive
computing, and covers everything from home appliance networks, to wireless
sensor networks and internet connected vehicles. At a later stage of the
project, the scope may be reduced to just one of these areas.  

\end{document}
